\documentclass[a4paper,14pt]{article}
\usepackage[14pt]{extsizes}
\usepackage{cmap}
\usepackage{color}
\usepackage{hyperref}
\usepackage{enumerate}
\usepackage[utf8]{inputenc} % указывает кодировку документа
\usepackage[T2A]{fontenc} % указывает внутреннюю кодировку TeX 
\usepackage[russian]{babel} % указывает язык документа  
%\usepackage{wasysym}  %for integrals?
%\usepackage{txfonts} %for integrals??
\usepackage{enumitem}
\usepackage{epstopdf}
\usepackage{graphicx}
%\graphicspath{{noiseimages/}}
\usepackage{amsmath}
\usepackage[left=20mm, top=15mm, right=15mm, bottom=15mm, nohead, footskip=10mm]{geometry} % настройки полей документа
  \renewcommand{\figurename}{Рисунок}


 
\RequirePackage{caption2} 
\renewcommand\captionlabeldelim{ -} 
 
 
 
\begin{document} % начало документа
\def\figurename{Рисунок}
 
 
 
 
% НАЧАЛО ТИТУЛЬНОГО ЛИСТА
\begin{center}
\hfill \break
%\large{МИНОБРНАУКИ РОССИИ}\\
%\footnotesize{ФЕДЕРАЛЬНОЕ ГОСУДАРСТВЕННОЕ БЮДЖЕТНОЕ ОБРАЗОВАТЕЛЬНОЕ УЧРЕЖДЕНИЕ}\\ 
%\footnotesize{ВЫСШЕГО ПРОФЕССИОНАЛЬНОГО ОБРАЗОВАНИЯ}\\
\small{\textbf{Липецкий государственный технический университет}}\\
\hfill \break
\normalsize{Институт машиностроения}\\
 \hfill \break
\normalsize{Кафедра оборудования и процессов машиностроительных производств}\\
\hfill\break
\hfill \break
\hfill \break
\hfill \break
\large{Моделирование термических напряжений\\ в пакете Matlab}\\
\hfill \break
\hfill \break
\hfill \break
%\normalsize{Магистерская диссертация\\
\normalsize{Отчет по практике}\\

\hfill \break
%Направление  010100 Математика\\
\hfill \break
%Магистерская программа    Вещественный, комплексный и функциональный анализ}\\
\hfill \break
\hfill \break
\end{center}
 
%\normalsize{ \hspace{28pt} Допущено к защите в ГЭК  27.05.2015} \hfill \break
\hfill \break
 
\normalsize{ 
\begin{tabular}{cccc}
%Зав.кафедрой & \underline{\hspace{3cm}} &  д.физ.-мат.н.,  проф. & Е.М. Семёнов \\\\\\
\\ Студент & \underline{\hspace{3cm}} &  & Е.П. Трофимов \\  Группа М-МД-16  \\\\
Руководитель & \underline{\hspace{3cm}}& к.т.н., доцент.&  А.И. Володин \\\\
\end{tabular}
}\\
\hfill \break
\hfill \break
\hfill \break
\hfill \break
\hfill \break
\hfill \break
\hfill \break
\hfill \break
\hfill \break
\hfill \break
%\hfill \break
\begin{center} Липецк, 2017 \end{center}
\thispagestyle{empty} % выключаем отображение номера для этой страницы
 
% КОНЕЦ ТИТУЛЬНОГО ЛИСТА
 
\newpage
%\section{Задание кафедры}
% ЗАДАНИЕ КАФЕДРЫВ     
     
\newpage     
     
     
\tableofcontents % Вывод содержания
\newpage
 
\newpage


\section{Решение уравнения теплопроводности в среде Matlab}
 

 
 \begin{figure}[h]
\centering
\includegraphics[scale=1]{2_comand.png}
\caption{Диалоговое окно PDE Toolbox}
\end{figure} 

Далее задаются условия на границах. Для этого нужно нажать на значок границы области (Рисунок -\ref{111}) и выбрать границу после чего появляется диалоговое окно с выбором граничных условий. В этом диалогом окне можно выбрать граничные условия как первого рода (условия Дирихле), так и второго (условия Неймана). Условие Дирихле задают температуру на границе, которая постоянно поддерживается, а условия Неймана задают тепловой поток. 
 	После чего во вкладке PDE задаются основные свойства материала. После чего переходим во вкладку Solve, далее в Parameters, где задаются время нагрева, начальная температура тела (в кельвинах). После задания всех параметров для решения нажимаем на «равно» в командной строке.
 	
 \begin{figure}[h]
 \label{111}
\centering
\includegraphics[scale=0.8]{3.png}
\caption{Диалоговое окно задания граничных условий}
\end{figure} 


После чего появляется окошко с графическим распределением температуры, наподобие: 
\begin{figure}[h]
 \label{}
\centering
\includegraphics[scale=0.8]{4.png}
\caption{Визуализация решения уравнения теплопроводности в Matlab PDE Toolbox}
\end{figure} 


 Настройки графического интерфейса находятся на вкладке справа от «равно». Matlab позволяет отображать линии тока, градиенты и др.Во вкладке Solve есть функция Export Solution которая возвращает массив значений в рабочее пространство. Столбцы данного массива соответствуют интервалам времени. Еще раз следует сказать о том, что PDE Toolbox Matlab позволяет задавать  произвольную двумерную форму для расчетов теплопроводности с граничными условиями первого и второго рода. Для задания третьего рода необходимо программирование численных методов, позволяющих найти температуру или напряжение. После настройки всех параметров можно получить следующее:
 
 
  \begin{figure}[h]
 \label{222}
\centering
\includegraphics[scale=0.8]{5.png}
\caption{Визуализация решения уравнения теплопроводности }
\end{figure} 


\section{Вывод}


В работе представлены теоритеческие основы термоупругости и термических напряжений. Данная методика является общей, охватывающей все случаи. При этом для каждой конкретной ситуации возможны упрощения и понижение сложности решаемой задачи (например, сведение трехмерной задачи к двумерной и т.д.). Для определения термомеханических напряжений первоочередной задачей является нахождение температурного поля в исследуемой детали. Для нахождения температурного поля могут использоваться различные математические пакеты. В данной работе мы рассмотрели такое средство как PDE Toolbox Matlab. Он очень удобен при решении двумерных задач, позволяя определять температурные поля для объектов произвольной формы с граничными условиями первого и второго рода. При получении заготовки свободной ковкой или горячей штампов­кой остаточные напряжения возникают из-за неравномерного охлажде­ния заготовок и особенно сильно сказываются при нерациональной кон­структивной форме последних. Эти напряжения имеют большое влияние на деформацию неустойчивых, маложестких заготовок (длинные валики, коленчатые валы и пр.). Следовательно, для минимазации данных напряжений, необходимо правильно расчитать напряжения возникающих в тепловых процессах.
При определенных условиях трехмерную деталь можно свести к двумерной модели (поля температур и напряжений будут одинаковы по всем сечениям детали). При этом моделирование теплопровых процессов в данном сечении является первостепенной задачей, один из способов решения которых, а именно: моделирования в пакете PDE Toolbox Matlab, был исследован в данной работе.

\newpage
\bibliographystyle{IEEEtran}
%\bibliography{<your-bib-database>}

%% Authors are advised to use a BibTeX database file for their reference list.
%% The provided style IEEEtran.bst formats references is generally used.

%% For references without a BibTeX database:

 \begin{thebibliography}{4}

%% \bibitem must have the following form:
%%   \bibitem{key}...
%%

 \bibitem{1} Анурьев В. И., \textsl{Справочник конструктора-машиностроителя}, Москва: Машиностроение, 2001. -- 920 стр.

 \bibitem{2} БанкетовА. Н., \textsl{Кузнечно-штаповочное оборудование: Учебник для машиностроительных вузов}, Москва: Машиностроение, 1982, -- 512 стр.

 \bibitem{3} Огаджанян О.И.,\textsl{ Расчёт кривошипных прессов}, Липецк, 2014г -- 77 с.

 \bibitem{4} Володин А. И., Володин И.М.\textsl{ Моделирование процессов горячей объемной штамповки} Москва: Машиностроение-1. 2006, -- 253 с.

 \end{thebibliography}


\end{document}  % КОНЕЦ ДОКУМЕНТА !

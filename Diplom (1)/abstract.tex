\newpage
\pagestyle{plain}
\setcounter{page}{3}
\section*{АННОТАЦИЯ}

	% С. ?. Ил. ?. Табл. ?. Литература ? назв. Прил. ?

	% В выпускной квалификационной работе проводится развитие методов графоструктурного моделирования. 

	% Развиваются матричные представления обобщенных графовых структур, в числе которых: графы, гиперграфы, сети, гиперсети и метаграфы.
	% Рассматриваются практические аспекты графоструктурного моделирования в задачах описания и анализа больших данных. Описать теоретико-множественный и матричный варианты представления графовых структур в контексте оптимизации вычислений в задачах описания и анализа больших данных, выделив недостатки и преимущества данных подходов. На этом основании показать обоснованность мотивации использования обобщённых графовых структур в таких задачах [5, 6]. Разработан и представлен алгоритм преобразования произвольного графа в метаграф, использующий матричноепредставление.

	% Для решения обозначенных в работе задач разработано программное обеспечение, предназначенное для предварительной обработки больших данных и основанное на гибридных интеллектуальных информационных системах, которые смоделированы графоструктурными методами.

	% \begin{table}[h]

	% \begin{tabular}{p{14.5cm} p{1cm}}
	% \multicolumn{2}{c}{ГРАФИЧЕСКАЯ ЧАСТЬ} \\
	% 	Слайд 1. Цель и задачи исследования  & 1 \\
	% 	Слайды 2-3. Подходы к вычислению...  & 2 \\
	% 	Слайд 4. Пример & 1 \\
	% \hline
	% 	Всего слайдов & 10
	% \end{tabular}

	% \end{table} 
